%%%%%%%%%%%%% preamble for all tex documents  %%%%
%%%%%%%%%%%%% current style defaults to scrartcl  but that could be changed
%%%%%%%%%%%%% with a script or by hand   
%%%%%%%%%%%%% 
\documentclass[10pt,notitlepage,abstracton]{scrartcl}
\usepackage[a4, center, dvips]{crop}
\setlength{\linewidth}{\textwidth}
\usepackage{scrpage2}
\usepackage{hhline}
\usepackage[sort,numbers]{natbib}
\usepackage{amsmath}
\usepackage{amsthm}
\usepackage{amsfonts}
\usepackage{amsbsy}
\usepackage{amsxtra}
\usepackage{amssymb}
\usepackage{scalerel}
\usepackage{calc}
\usepackage[draft]{fixme}
\usepackage[utf8]{inputenc}
\usepackage{float}
\usepackage{url}
\usepackage{graphicx}
\usepackage{color}
\usepackage[ruled, section]{algorithm}
\usepackage{algorithmicx}
\usepackage{algpseudocode}
\usepackage{multirow}
\usepackage{tabularx}
\usepackage{subfigure}
\usepackage[normalem]{ulem}
\theoremstyle{plain}
\newtheorem{theorem}{Theorem}[section]
\newtheorem{lemma}[theorem]{Lemma}
\newtheorem{proposition}[theorem]{Proposition}
\newtheorem{corollary}[theorem]{Corollary}
\theoremstyle{plain}
\newtheorem{definition}{Definition}[section]
\newtheorem{example}{Example}[section]
\theoremstyle{plain}
\newtheorem*{remark}{Remark}
\newtheorem*{note}{Note}
\newtheorem{case}{Case}
\makeatletter
\renewcommand{\labelitemi}{$\m@th\triangleright$}
\renewcommand{\labelitemii}{$\m@th\cdot$}
\renewcommand{\labelitemiii}{\emdash}
\renewcommand{\labelitemiv}{\emdash\emdash}
\makeatother

\makeatletter
\newsavebox{\@brx}
\newcommand{\llangle}[1][]{\savebox{\@brx}{\(\m@th{#1\langle}\)}%
  \mathopen{\copy\@brx\kern-0.5\wd\@brx\usebox{\@brx}}}
\newcommand{\rrangle}[1][]{\savebox{\@brx}{\(\m@th{#1\rangle}\)}%
  \mathclose{\copy\@brx\kern-0.5\wd\@brx\usebox{\@brx}}}
\makeatother
\newcommand\reallywidehat[1]{\arraycolsep=0pt\relax%
  \begin{array}{c}
    \stretchto{
    \scaleto{
    \scalerel*[\widthof{\ensuremath{#1}}]{\kern-.5pt\bigwedge\kern-.5pt}
    {\rule[-\textheight/2]{1ex}{\textheight}} %WIDTH-LIMITED BIG WEDGE
    }{\textheight} % 
    }{0.5ex}\\           % THIS SQUEEZES THE WEDGE TO 0.5ex HEIGHT
    #1\\                 % THIS STACKS THE WEDGE ATOP THE ARGUMENT
    \rule{-1ex}{0ex}
  \end{array}
}


\include{notation-macros}
\usepackage[scrpage]{svninfo} 
%%%% not sure what to make of revision info right now
\title{Stable simulation of a lumped element rod}
\author{Claude Lacoursi{\`{e}}re \\
  HPC2N/UMIT \\
  and \\
  Department of Computing Science \\
  Ume{\aa} University\\
  SE-901 87, Ume{\aa}, Sweden\\
  \texttt{claude@hpc2n.umu.se}
}
\begin{document}
\svnInfo $Id: contact.tex 768 2016-04-04 07:48:45Z claude $
\maketitle{}
\svnId
\begin{abstract}

\end{abstract}

\section{Introduction}
\label{sec:introduction}



\section{Lumped element model}
\label{sec:optim-cont-point}

We consider a one dimensional rod with $N$ elements of identical mass $m$
and coordinate $\theta$, coupled with spring-damper forces
\begin{equation}
  \label{eq:1}
  \idx{\tau}{i,i+1} = -K(\idx{\theta}{i}-\idx{\theta}{i+1}) -
  \gamma(\idx{\omega}{i}-\idx{\omega}{i+1}), \text{ and } \idx{\tau}{i+1,i}
  = -
  \idx{\tau}{i,i+1}. 
\end{equation}
We are particularly interested in the case where $K$ is arbitrarily large,
even including $K=\infty$, which justifies the choice of implicit
integration below.

The rod can also be forced externally with $\idx{\tau}{1}$ and $\idx{\tau}{N}$.
In addition to this, it can be subjected to force-velocity coupling with
external dynamical systems, assuming these are physical.  This is done by
adding input spring-dampers at either end which produce forces according to 
\begin{equation}
  \label{eq:2}
  \idx{\tau}{c,i} = 
-\idx{K}{c}(\idx{\theta}{i}- \idx{\bar{\theta}}{c,i})
-\idx{\gamma}{c}(\idx{\omega}{i}- \idx{\bar{\omega}}{c,i})
\end{equation}
where $(\bar{\cdot})$ are either last known quantities, or extrapolations.
We also consider the case where only the velocity $\idx{\bar{\omega}}{c,i}$
is known in which case we have
\begin{equation}
  \label{eq:3}
  \begin{split}
    \idx{\tau}{c,i} &= 
    -\idx{K}{c}\idx{\delta\theta}{i}
    -\idx{\gamma}{c}(\idx{\omega}{i}- \idx{\bar{\omega}}{c,i})\text{ with } \\
    \idx{\dot{\delta\theta}}{i} & =
    \idx{\omega}{i}-\idx{\bar{\omega}}{e,i}. 
  \end{split}
\end{equation}
This coupling force is reported to the coupled element of course.  Given
the input and output signals here, this rod can be coupled to
other subsystems either kinematically, or via force-velocity or even
force-displacements couplings.

Given stiffness requirements, we use a stable
integrator, namely, \textsc{Spook}.  To get maximum stability, we first
introduce the matrix  $\bar{G}$ which contains internal, pairwise
interactions as 
\begin{equation}
  \label{eq:15}
  \begin{split}
    \bar{G} & =
    \begin{bmatrix}
      1 &  -1  & 0 & 0 &  \cdots \\
      0 &   \phantom{-}1  & -1 & 0 &  \cdots \\
      \vdots &   0  &  \ddots & \ddots &  \ddots \\
      \vdots &   \vdots  &  \vdots &   1 & -1  \\
    \end{bmatrix}. 
  \end{split}
\end{equation}
Given its form, the implication here is that $G\theta = 0$ is the unique
rest configuration, where
\begin{equation}
  \label{eq:6}
  \theta =
  \begin{bmatrix}
    \idx{\theta}{1},\idx{\theta}{2}, \ldots, \idx{\theta}{n}
  \end{bmatrix}^{T}. 
\end{equation}
Henceforth, quantities without superscripts are understood to be vectors or
matrices.  We can therefore associate a potential energy
\begin{equation}
  \label{eq:4}
  U  = \frac{1}{2}\theta^{T} \bar{G}^{T}K\bar{G}\theta. 
\end{equation}
If we include the external forces due drivers, again, anticipating
very high stiffness, we need two additional rows so that
\begin{equation}
  \label{eq:32}
    G = \begin{bmatrix}
      1 &  0  & 0 & 0 &  \cdots \\
      1 &  -1  & 0 & 0 &  \cdots \\
      0 &   \phantom{-}1  & -1 & 0 &  \cdots \\
      \vdots &   0  &  \ddots & \ddots &  \ddots \\
      \vdots &   \vdots  &  \ddots &   1 & -1  \\
      \vdots &   \vdots  &  \vdots &   0 & \phantom{-}1
    \end{bmatrix}. 
\end{equation}
Under infinite stiffness, these additional rows impose the constraint
\begin{equation}
  \label{eq:5}
\idx{\omega}{i} = \idx{\bar{\omega}}{e,i}. 
\end{equation}
\begin{anfxnote}{no nonholonomic coupling now}
  The form of the matrix in Eqn.~\eqref{eq:32} isn't implemented as an
  option yet meaning that it isn't possible to drive the rod with a hard
  kinematic driver.  The reason being that we would need a special flag to
  choose which form the system matrix takes.  
\end{anfxnote}
This can also be modified to include the case of displacement-displacement
coupling as described further below, as well as the case of high but finite
stiffness for coupling. 

After a Legendre transform and discretization with fixed step $h$ and
discrete time $k$ we arrive at 
\begin{equation}
  \label{eq:18}
  \begin{split}
    \begin{bmatrix}
      \mathcal{J} & -G^{T} \\
      G & \tilde{K}^{-1}
    \end{bmatrix}
    \begin{bmatrix}
      \omega_{k+1} \\
      \lambda
    \end{bmatrix}
    &=
    \begin{bmatrix}
      \mathcal{J}\omega_{k} + h\tau \\
      -\frac{4}{h}\gamma G\theta_{k}  + \gamma G\omega_{k}
    \end{bmatrix} \\
    \theta_{k+1} &= \theta_{k} + h\omega_{k}
  \end{split}
\end{equation}
where $\mathcal{J}$ is the (diagonal) inertia matrix and 
\begin{equation}
  \label{eq:19}
  \begin{split}
    \gamma &= \frac{1}{1 + 4 \rho / h } \text{ and $\rho$ is a relaxation
      rate } \\
    \tilde{K} & = \frac{h^{2}\idx{K}{i}}{4\gamma}
  \end{split}
\end{equation}
The $\rho$ parameter has units of inverse time and describes a relaxation
rate, i.e., damping, and $\rho/h$ is the number of steps it takes to loose
half the excitation amplitude locally.  When this is set to $\rho/h= 2$,
the energy dissipates rapidly, and this provides unconditional stability
via numerical damping.
However, as soon as we reach $\rho/h < 1/10$ or so, we have natural
damping.  

We can of course eliminate the Lagrange multipliers $\lambda$ and solve
instead 
\begin{equation}
  \label{eq:22}
  \left[ \mathcal{J} + G^{T}\tilde{K}G \right]\omega_{k+1} =
  \mathcal{J}\omega_{k} + h\tau -
  \frac{4\gamma}{h}G^{T}\tilde{K}G\theta_{k} + \gamma G^{T}\tilde{K}G\omega_{k}. 
\end{equation}
where $\tilde{K}$ is defined as before in Eqn.~\eqref{eq:19}.  This
requires finite $K$ however.  But we can mix the two methods and treat the
external forcing terms using the relevant part of Eqn.~\eqref{eq:22} and
the rest with Eqn.~\eqref{eq:18}.  

\section{Coupling details}
\label{sec:coupling-details}

Coupling to other simulations comes in various forms.  A pure torque
coupling simply adds $h\idx{\tau}{i,c}$ to the RHS in  Eqn.~\eqref{eq:18}.
A velocity or displacement coupling adds
$\frac{h^{2}}{4\gamma}\idx{K}{i,c}$ to the diagonal as well as
\begin{equation}
  \label{eq:7}
  h\idx{\tau}{i,c} = -h\idx{K}{i,c}(\idx{\theta}{i}-\idx{\bar{\theta}}{i,c}) +
  \frac{h^{2}}{4}(\idx{\omega}{i}-\idx{\bar{\omega}}{i})
\end{equation}
or
\begin{equation}
  \label{eq:8}
  h\idx{\tau}{i,c} = -\idx{K}{i,c}\delta\idx{\theta}{i} + \frac{h^{2}}{4}(\idx{\omega}{i}-\idx{\bar{\omega}}{i})
\end{equation}
on the RHS of Eqn.~\eqref{eq:18}.     For the latter case, we update
$\delta\idx{\theta}{i}$ with
\begin{equation}
  \label{eq:10}
  \delta\idxsub{\theta}{i}{k+1} = \delta\idxsub{\theta}{i}{k} +
  h(\idxsub{\omega}{i}{k+1}-\idx{\bar{\omega}}{i,c}).
\end{equation}



\renewcommand{\arraystretch}{1.5}

Output variables. 

\begin{tabularx}{1.0\linewidth}[H]{ ||X||c|c|c|c|c|| }
  \hhline{|======|}
  Name & Symbol & Variable & Value ref.  & Default & Causality\\
  \hhline{|======|}
  First angle & $\theta_{1}$ &\texttt{theta1} & 0 &&\\ \hline
  Last angle & $\theta_{2}$ &\texttt{theta2} & 1 &&\\ \hline
  First angular speed & $\omega_{1}$ &\texttt{omega1} & 2 &&\\ \hline
  Last angular speed & $\omega_{2}$ &\texttt{omega2} & 3 &&\\ \hline
  First angular acceleration & $\alpha_{1}$ &\texttt{alpha1} & 4 &&\\ \hline
  Last angular acceleration & $\alpha_{2}$ &\texttt{alpha2} & 5 &&\\ \hline
  First angle difference& $\delta\theta_{1}$ &\texttt{dtheta1} & 6 &&\\ \hline
  Last angle difference& $\delta\theta_{2}$ &\texttt{dtheta2} & 7 &&\\ \hline
  First output torque  & $\idx{\tau}{c,1}$ &\texttt{out\_torque1} & 8 &&\\ \hline
  Last output torque  & $\idx{\tau}{c,N}$ &\texttt{out\_torque2} & 9 &&\\ \hline
\hhline{|======|}
\end{tabularx}

Inputs.

\begin{tabularx}{1.0\linewidth}[H]{ ||X||c|c|c|c|c|| }
  \hhline{|======|}
  Name & Symbol & Variable & Value ref.  & Default & Causality\\
  \hhline{|======|}
  First input torque  & $\idx{\tau}{1}$ &\texttt{tau1} & 10 &&\\ \hline
  Last input torque  & $\idx{\tau}{N}$ &\texttt{tau2} & 11 &&\\ \hline
  First driving displacement & $\idx{\bar{\theta}}{c,1}$
                &\texttt{theta\_drive1} & 12 &&\\ \hline
  First driving velocity & $\idx{\bar{\omega}}{c,1}$
                &\texttt{omega\_drive1} & 13 &&\\ \hline
  Last driving displacement & $\idx{\bar{\theta}}{c,N}$
                &\texttt{theta\_drive2} & 14 &&\\ \hline
  Last driving velocity & $\idx{\bar{\omega}}{c,N}$
                &\texttt{omega\_drive2} & 15 &&\\ \hline
  \hhline{|======|}
\end{tabularx}

Parameters.

\begin{tabularx}{1.0\linewidth}[H]{ ||X||c|c|c|c|c|| }
  \hhline{|======|}
  Name & Symbol & Variable & Value ref.  & Default & Causality\\
  \hhline{|======|}
  Moment of inertia & $\mathcal{J}$
                &\texttt{J} & 16 &&\\ \hline
  Compliance & $K^{-1}$
                &\texttt{compliance} & 17 &&\\ \hline
  Damping rate & $\rho/h$
                &\texttt{D} & 18 &&\\ \hline
  First driving spring & $\idx{K}{c,1}$
                &\texttt{K\_drive1} & 19 &&\\ \hline
  First driving damping & $\idx{\gamma}{c,1}$
                &\texttt{D\_drive1} & 20 &&\\ \hline
  Last driving spring & $\idx{K}{c,2}$
                &\texttt{K\_drive2} & 21 &&\\ \hline
  Last driving damping & $\idx{\gamma}{c,2}$
                &\texttt{D\_drive2} & 22 &&\\ \hline
  Time step & $h$
                &\texttt{step} & 23 &&\\ \hline
  Number of elements & $N$
                &\texttt{n\_elements} & 24 &&\\ \hline
  First initial angle & $\idxsub{\theta}{1}{0}$
                &\texttt{theta01} & 25 &&\\ \hline
  Last initial angle & $\idxsub{\theta}{N}{0}$
                &\texttt{theta02} & 26 &&\\ \hline
  First initial angular speed & $\idxsub{\omega}{1}{0}$
                &\texttt{omega01} & 27 &&\\ \hline
  Last initial angular speed & $\idxsub{\omega}{2}{0}$
                &\texttt{omega02} & 28 &&\\ \hline
  Sign for the first driver & $\idx{\sigma}{1}$
                &\texttt{driver\_sign1} & 29 &&\\ \hline
  Sign for the last driver & $\idx{\sigma}{N}$
                &\texttt{driver\_sign2} & 30 &&\\ \hline
  Integrate angle difference $1$& 
                &\texttt{integrate\_dt1} & 31 &&\\ \hline
  Integrate angle difference $N$& 
                &\texttt{integrate\_dt2} & 32 &&\\ \hline
\hhline{|======|}
\end{tabularx}
\section{Conclusion}

\label{sec:conclusion}

\section*{Acknowledgments}
\label{sec:acknowledgments}


\bibliographystyle{plainnat}  
\bibliography{lde-bib}
\end{document}

%%% Local Variables: 
%%% mode: latex
%%% TeX-master: t
%%% End: 
