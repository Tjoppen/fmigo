%%%%%%%%%%%%% preamble for all tex documents  %%%%
%%%%%%%%%%%%% current style defaults to scrartcl  but that could be changed
%%%%%%%%%%%%% with a script or by hand   
%%%%%%%%%%%%% 
\documentclass[10pt,notitlepage,abstracton]{scrartcl}
\usepackage[a4, center, dvips]{crop}
\setlength{\linewidth}{\textwidth}
\usepackage{scrpage2}
\usepackage{hhline}
\usepackage[sort,numbers]{natbib}
\usepackage{amsmath}
\usepackage{amscd}
\usepackage{amsthm}
\usepackage{amsfonts}
\usepackage{amsbsy}
\usepackage{amsxtra}
\usepackage{amssymb}
\usepackage[draft]{fixme}
\usepackage[utf8]{inputenc}
\usepackage{float}
\usepackage{url}
\usepackage{graphicx}
\usepackage{color}
\usepackage[ruled, section]{algorithm}
\usepackage{algpseudocode}
\usepackage{multirow}
\usepackage{tabularx}
\usepackage{subfigure}
\usepackage{tikz}
\usepackage{pgfplots,pgfplotstable}
\usepackage{gnuplot-lua-tikz}
\usepackage[normalem]{ulem}
\theoremstyle{plain}
\newtheorem{theorem}{Theorem}[section]
\newtheorem{lemma}[theorem]{Lemma}
\newtheorem{proposition}[theorem]{Proposition}
\newtheorem{corollary}[theorem]{Corollary}
\theoremstyle{plain}
\newtheorem{definition}{Definition}[section]
\newtheorem{example}{Example}[section]
\theoremstyle{plain}
\newtheorem*{remark}{Remark}
\newtheorem*{note}{Note}
\newtheorem{case}{Case}
\makeatletter
\renewcommand{\labelitemi}{$\m@th\triangleright$}
\renewcommand{\labelitemii}{$\m@th\cdot$}
\renewcommand{\labelitemiii}{\emdash}
\renewcommand{\labelitemiv}{\emdash\emdash}
\makeatother
\include{notation-macros}
\usepackage[scrpage]{svninfo} 
%%%% not sure what to make of revision info right now
\title{Trailer model}
\author{Claude Lacoursi{\`{e}}re \\
  HPC2N/UMIT, Ume{\aa} University\\
  SE-901 87, Ume{\aa}, Sweden\\
  \texttt{claude@hpc2n.umu.se}
}
\begin{document}
\svnInfo $Id$
\maketitle{}
\svnId
\begin{abstract}

\end{abstract}

\section{Introduction}
\label{sec:introduction}

This is a simple model for a ``trailer'', a point mass moving along a
straight line but with variable elevation.   This is intended to perform
simple stability tests for cosimulation algorithms, especially for the case
where different simulations involve objects of different masses. 

\section{The model}
\label{sec:model}

We represent the trailer as a point mass $m$ moving in a straight line. Even
though we are considering the trailer to have both vertical and
longitudinal motion, we take $x\in\R$ as the distance traveled along the road
which has variable grade $\theta(x)$.  


The trailer is powered via a shaft with angle and angular speed
$\phi_{d},\omega_{d}$.  This shaft goes is connected to the axle
with a gear with ratio $r_{d}$ from the differential.  Assuming no
slip, the speed of the point mass is related to that of the wheel via
$\dot{x} = r_{w}\omega_{w}$, and this corresponds also to the angular speed
of the differential shaft, $\omega_{s} = \dot{x}/(r_{w}r_{d})$.

The relevant portion of the gravity force is
\begin{equation}
  \label{eq:1}
  f_{g} = - m g \sin(\theta ( x ) ).
\end{equation}
A drag force for air resistance is introduced as
\begin{equation}
  \label{eq:2}
  f_{d} = - \sign(\dot{x}) \frac{1}{2}\rho A C_{d} \dot{x}^{2},  
\end{equation}
rolling resistance is then
\begin{equation}
  \label{eq:3}
  f_{r} = - \sign(\dot{x})(c_{r_{1}}\abs{\dot{x}} + c_{r_{0}}) m g \cos(\theta(x))
\end{equation}
and a braking force
\begin{equation}
  \label{eq:4}
  f_{b} = - r \sign(\dot{x})\mu m g \cos(\theta(x)), \text{ where }   r\in[0,1].
\end{equation}
As for the input forces, we either accept a torque directly in which case
\begin{equation}
  \label{eq:5}
  f_{c} = \frac{1}{r_{w}r_{d}}\tau_{i}
\end{equation}
or we use a coupling spring with
\begin{equation}
  \label{eq:6}
  \begin{split}
    f_{c} &= \frac{1}{r_{w}} \left(  k_{d} \delta 
      +\gamma_{d}( \frac{\dot{x}}{r_{d}r_{w}} - \bar{\omega}_{d} ) \right) \\
    \dot{\delta} &= \frac{\dot{x}}{r_{d}r_{w}} - \bar{\omega}_{d}. 
  \end{split}
\end{equation}
For the case of displacement-displacement coupling, we use 
\begin{equation}
  \label{eq:8}
    f_{c} = \frac{1}{r_{w}} \left( k_{d} (\frac{x}{r_{d}r_{w}} - \bar{\phi}_{d})  
      +\gamma_{d}( \frac{\dot{x}}{r_{d}r_{w}} - \bar{\omega}_{d} ) \right) 
\end{equation}
We also include a coupling directly on the trailer and this is computed
either as
\begin{equation}
  \label{eq:9}
  \begin{split}
    f_{t} &=  k_{t} \delta x + \gamma_{t}(\dot{x} - v_{t}) \\
    \delta\dot{x} &= \dot{x} - v_{t}
  \end{split}
\end{equation}
or
\begin{equation}
  \label{eq:10}
  f_{t}  = k_{t}(x-x_{t}) + \gamma( \dot{x} - v_{t}).
\end{equation}

The equations motion are then, allowing for one further external force $f_{e}$, 
\begin{equation}
  \label{eq:7}
  m\ddot{x} = f_{d} + f_{g} + f_{r} + f_{b} - f_{c} + f_{e} - f_{t}. 
\end{equation}


\section{Parameters, inputs and outputs}
\label{sec:param-inputs-outp}

The list of parameters is as follows.  Note that the default values are
worthless but they have been chosen so the module does not crash if the
user forgets to set them.  All units are assumed to be in MKS units. 


\renewcommand{\arraystretch}{1.5}

\begin{tabularx}{1.0\linewidth}[H]{ ||X|c|c|c|c|c|| }
  \hhline{|======|}
  Name & Symbol & Variable & Value ref. & Range & Default \\
  \hhline{|======|}
  Initial position & $x_{0}$ & \texttt{x\_0} & 1  & $(-\infty,\infty)$ & 0 \\ \hline
  Initial velocity & $v_{0}$ & \texttt{v\_0} & 2  & $(-\infty,\infty)$ & 0 \\ \hline
  Mass  & $m$ & \texttt{mass}& 3 & $(0,\infty]$ & 1 Kg \\ \hline
  Wheel radius &$r_{w}$& \texttt{r\_w} & 4 &  $(0,\infty]$  &1 m \\ \hline
  Differential gear ratio & $r_{g}$ &\texttt{r\_g} & 5 & $(0,\infty]$ & 1 \\ \hline
  Surface area & $A$& \texttt{area} & 6  & $[0,\infty]$ & $1\text{m}^{2}$ \\ \hline
  Air density & $\rho$ & \texttt{rho} & 7 & $(0,\infty]$ & $1.2$Kg/$\text{m}^{3}$ \\ \hline
  Drag coefficient & $C_{d}$ & \texttt{c\_d} & 8 & $[0,\infty]$ & $1$ \\ \hline
  Gravitational \newline acceleration & $g$ &\texttt{g} & 9 & $(-\infty, \infty)$ & $9.8$ m/$\text{s}^{2}$ \\ \hline
  Static rolling \newline resistance & $c_{r_{1}}$ &\texttt{c\_r\_1} & 10 & $[0, \infty)$ & ???  \\ \hline
  Kinetic rolling \newline resistance & $c_{r_{2}}$ &\texttt{c\_r\_2} & 11 & $[0, \infty)$ & ??  \\ \hline
  Static tire 
  friction & $\mu$ & \texttt{mu} & 12 & $[0, \infty)$ & 1 \\ \hline
  Differential coupling spring constant & $k_{d}$ &\texttt{k\_d} & 13 & $[0,\infty]$ & \\ \hline
  Differential  coupling damping  constant & $\gamma$
                &\texttt{gamma\_d} & 14 & $[0,\infty]$ & N$\cdot$ s/m \\\hline 
  Trailer coupling spring  constant & $k_{t}$ &\texttt{k\_t} & 15 & $[0,\infty]$ & \\ \hline
  Trailer coupling damping  constant & $\gamma_{t}$
                &\texttt{gamma\_t} & 16 & $[0,\infty]$ & N$\cdot$ s/m \\\hline 
  Integrate angle &  & \texttt{\footnotesize{integrate\_dw}} & 17 &  & integer \\\hline
  Integrate position &  & \texttt{\footnotesize{integrate\_dx}} & 18 &  & integer \\\hline
  Integrator &  & & 12 &  & integer \\\hline
  EPCE filter &  & \texttt{\footnotesize{filter\_length}} & 98 & $[0,1,3,\ldots]$ & 0 steps \\ \hhline{|======|}
\end{tabularx}


The initial values which must be set. 



Output variables. 

\begin{tabularx}{1.0\linewidth}[H]{ ||X||c|c|c|| }
  \hhline{|====|}
  Name & Symbol & Variable & Value ref.   \\
  \hhline{|====|}
  Position &  $x$ & \texttt{x} & 29 \\ \hline
  Velocity &  $v=\dot{x}$ & \texttt{v} & 30 \\ \hline
  Acceleration &  $a=\ddot{x}$ & \texttt{a} & 31 \\ \hline
  Shaft angle &  $\phi = \frac{x}{r_{w}r_{g}}$ & \texttt{phi} & 32 \\
  \hline
  Shaft speed &  $\omega = \frac{\dot{x}}{r_{w}r_{g}}$ & \texttt{omega} & 33 \\ \hline
  Shaft acceleration &  $\alpha = \frac{\ddot{x}}{r_{w}r_{g}}$ & \texttt{alpha} & 34 \\ \hline
  Coupling torque  & $\tau_{c}$ &\texttt{tau\_c} & 35 \\ \hline
  Coupling force  & $f_{c}$ &\texttt{f\_c} & 36 \\ \hhline{|====|}
\end{tabularx}

Input variables.

\begin{tabularx}{1.0\linewidth}[H]{ ||X|c|c|c|| }
  \hhline{|====|}
  Name & Symbol & Variable & Value ref. \\
  \hhline{|====|}
  Input angle &  $\phi_{i}$ & \texttt{phi\_i} & 20\\ \hline
  Input velocity &  $\omega_{i}$ & \texttt{omega\_i} & 21 \\ \hline
  Coupling torque &  $\tau_{d}$ & \texttt{tau\_d} &  22 \\ \hline
  Extra driving torque &  $\tau_{e}$ & \texttt{tau\_e} & 23 \\ \hline
  Slope & $\theta$ &\texttt{angle} & 24 \\ \hline
  Brake position & $r$ & \texttt{brake}& 25 \\ \hline
  Coupling displacement & $x_{in}$ & \texttt{x\_in}& 26 \\ \hline
  Coupling speed & $v_{in}$ & \texttt{v\_in}& 27 \\ \hline
  Coupling force & $f_{in}$ & \texttt{f\_in}& 28 \\ 
 \hhline{|====|}
\end{tabularx}









\section{Conclusion}
\label{sec:conclusion}

\section*{Acknowledgments}
\label{sec:acknowledgments}


\bibliographystyle{plainnat}  
\bibliography{lde-bib}
\end{document}

%%% Local Variables: 
%%% mode: latex
%%% TeX-master: t
%%% End: 
