%%%%%%%%%%%%% preamble for all tex documents  %%%%
%%%%%%%%%%%%% current style defaults to scrartcl  but that could be changed
%%%%%%%%%%%%% with a script or by hand   
%%%%%%%%%%%%% 
\documentclass[10pt,notitlepage,abstracton]{scrartcl}
\usepackage[a4, center, dvips]{crop}
\setlength{\linewidth}{\textwidth}
\usepackage{scrpage2}
\usepackage{hhline}
\usepackage[sort,numbers]{natbib}
\usepackage{amsmath}
\usepackage{amscd}
\usepackage{amsthm}
\usepackage{amsfonts}
\usepackage{amsbsy}
\usepackage{amsxtra}
\usepackage{amssymb}
\usepackage[draft]{fixme}
\usepackage[utf8]{inputenc}
\usepackage{float}
\usepackage{url}
\usepackage{graphicx}
\usepackage{color}
\usepackage[ruled, section]{algorithm}
\usepackage{algpseudocode}
\usepackage{multirow}
\usepackage{tabularx}
\usepackage{subfigure}
\usepackage{tikz}
\usepackage{pgfplots,pgfplotstable}
\usepackage{gnuplot-lua-tikz}
\usepackage[normalem]{ulem}
\theoremstyle{plain}
\newtheorem{theorem}{Theorem}[section]
\newtheorem{lemma}[theorem]{Lemma}
\newtheorem{proposition}[theorem]{Proposition}
\newtheorem{corollary}[theorem]{Corollary}
\theoremstyle{plain}
\newtheorem{definition}{Definition}[section]
\newtheorem{example}{Example}[section]
\theoremstyle{plain}
\newtheorem*{remark}{Remark}
\newtheorem*{note}{Note}
\newtheorem{case}{Case}
\makeatletter
\renewcommand{\labelitemi}{$\m@th\triangleright$}
\renewcommand{\labelitemii}{$\m@th\cdot$}
\renewcommand{\labelitemiii}{\emdash}
\renewcommand{\labelitemiv}{\emdash\emdash}
\makeatother
\include{notation-macros}
\usepackage[scrpage]{svninfo} 
%%%% not sure what to make of revision info right now
\title{Trailer model}
\author{Claude Lacoursi{\`{e}}re \\
  HPC2N/UMIT, Ume{\aa} University\\
  SE-901 87, Ume{\aa}, Sweden\\
  \texttt{claude@hpc2n.umu.se}
}
\begin{document}
\svnInfo $Id$
\maketitle{}
\svnId
\begin{abstract}

\end{abstract}

\section{Introduction}
\label{sec:introduction}

This is a simple model for a ``trailer'', a point mass moving along a
straight line but with variable elevation.   This is intended to perform
simple stability tests for cosimulation algorithms, especially for the case
where different simulations involve objects of different masses. 

\section{The model}
\label{sec:model}

We represent the trailer as a point mass $m$ moving in a straight line. Even
though we are considering the trailer to have both vertical and
longitudinal motion, we take $x\in\R$ as the distance traveled along the road
which has variable grade $\alpha(x)$.  


The trailer is powered via a shaft with angle and angular speed
$\idx{\phi}{i},\idx{\omega}{i}$.  This shaft goes is connected to the axle
with a gear with ratio $r_{d}$ from the differential.  Assuming no
slip, the speed of the point mass is related to that of the wheel via
$\dot{x} = r_{w}\idx{\omega}{w}$.

The relevant portion of the gravity force is
\begin{equation}
  \label{eq:1}
  f_{g} = - m g \sin(\alpha ( x ) ).
\end{equation}
A drag force for air resistance is introduced as
\begin{equation}
  \label{eq:2}
  f_{d} = - \sign(\dot{x}) \frac{1}{2}\rho A C_{d} \dot{x}^{2},  
\end{equation}
rolling resistance is then
\begin{equation}
  \label{eq:3}
  f_{r} = - \sign(\dot{x})(c_{r_{1}}\abs{\dot{x}} + c_{r_{0}}) m g \cos(\alpha(x))
\end{equation}
and a braking force
\begin{equation}
  \label{eq:4}
  f_{b} = - r \sign(\dot{x})\mu m g \cos(\alpha(x)), \text{ where }   r\in[0,1].
\end{equation}
As for the input forces, we either accept a torque directly in which case
\begin{equation}
  \label{eq:5}
  f_{c} = \frac{1}{r_{w}r_{d}}\tau_{i}
\end{equation}
or we use a coupling spring with
\begin{equation}
  \label{eq:6}
  \begin{split}
    f_{i} &= \frac{1}{r_{w}} \left( -k \delta -
      \gamma( \frac{\dot{x}}{r_{d}r_{w}} - \bar{\omega}_{i} ) \right) \\
    \dot{\delta} &= \frac{\dot{x}}{r_{d}r_{w}} - \bar{\omega}_{i}. 
  \end{split}
\end{equation}
For the case of displacement-displacement coupling, we use 
\begin{equation}
  \label{eq:8}
    f_{i} = \frac{1}{r_{w}} \left( -k (\frac{x}{r_{d}r_{w}} - \bar{\phi}_{i})  -
      \gamma( \frac{\dot{x}}{r_{d}r_{w}} - \bar{\omega}_{i} ) \right) 
\end{equation}



The equations motion are then, allowing for one further external force $f_{e}$, 
\begin{equation}
  \label{eq:7}
  m\ddot{x} = f_{d} + f_{g} + f_{r} + f_{b} + f_{c} + f_{e}. 
\end{equation}


\section{Parameters, inputs and outputs}
\label{sec:param-inputs-outp}

The list of parameters is as follows.  Note that the default values are
worthless but they have been chosen so the module does not crash if the
user forgets to set them.  All units are assumed to be in MKS units. 



\begin{tabularx}{1.0\linewidth}[H]{ ||X|c|c|c|c|c|| }
  \hhline{|======|}
  Name & Symbol & Variable & Value ref. & Range & Default \\
  \hhline{|======|}
  Initial position & $x_{0}$ & \texttt{x\_0} & 1  & $(-\infty,\infty)$ & 0 \\ \hline
  Initial velocity & $v_{0}$ & \texttt{v\_0} & 2  & $(-\infty,\infty)$ & 0 \\ \hline
  Mass  & $m$ & \texttt{mass}& 3 & $(0,\infty]$ & 1 Kg \\ \hline
  Wheel radius &$r_{w}$& x & 4 &  $(0,\infty]$  &1 m \\ \hline
  Differential gear ratio & $r_{g}$ &\texttt{r\_g} & 5 & $(0,\infty]$ & 1 \\ \hline
  Surface area & $A$& \texttt{area} & 6  & $[0,\infty]$ & $1\text{m}^{2}$ \\ \hline
  Air density & $\rho$ & \texttt{rho} & 7 & $(0,\infty]$ & $1.2$Kg/$\text{m}^{3}$ \\ \hline
  Drag coefficient & $C_{d}$ & \texttt{c\_d} & 8 & $[0,\infty]$ & $1$ \\ \hline
  Gravitational \newline acceleration & $g$ &\texttt{g} & 9 & $(-\infty, \infty)$ & $9.8$ m/$\text{s}^{2}$ \\ \hline
  Static rolling \newline resistance & $c_{r_{1}}$ &\texttt{c\_r\_1} & 10 & $[0, \infty)$ & ???  \\ \hline
  Kinetic rolling \newline resistance & $c_{r_{2}}$ &\texttt{c\_r\_2} & 11 & $[0, \infty)$ & ??  \\ \hline
  Static tire friction & $\mu$ & \texttt{mu} & 12 & $[0, \infty)$ & 1 \\ \hline
  Coupling spring \newline constant & $k$ &\texttt{k} & 13 & $[0,\infty]$ & ??? \\ \hline
  Coupling damping \newline constant & $\gamma$ &x & 12 & $[0,\infty]$ & N$\cdot$ s/m \\\hline 
  Integrator &  &x & 12 &  & integer \\
  EPCE filter &  & \texttt{filter\_length} & 98 & $[0,1,3,\ldots]$ & 0 steps \\ \hhline{|======|}
\end{tabularx}


The initial values which must be set. 



Output variables. 


\begin{tabularx}{1.0\linewidth}[H]{ ||X|c|c|c|c|| }
  \hhline{|=====|}
  Name & Symbol & Variable & Value ref. & Range  \\
  \hhline{|=====|}
  Position &  $x$ & \texttt{x} & 21 & \\ \hline
  Velocity &  $v=\dot{x}$ & \texttt{v} & 22 & \\ \hline
  Acceleration &  $a=\ddot{x}$ & \texttt{a} & 23 & \\ \hline
  Coupling force  & $f_{c}$ &\texttt{f\_c} & 25 &  \\ \hhline{|====|}
\end{tabularx}

Input variables.

\begin{tabularx}{1.0\linewidth}[H]{ ||X|c|c|c|c|| }
  \hhline{|=====|}
  Name & Symbol & Variable & Value ref. & Range  \\
  \hhline{|=====|}
  Input angle &  $\phi_{i}$ & \texttt{phi\_i} & 16 & \\ \hline
  Input velocity &  $\omega_{i}$ & \texttt{omega\_i} & 17 & \\ \hline
  Driving in torque &  $\tau_{d}$ & \texttt{tau\_d} & 18 & \\ \hline
  Extra driving torque &  $\tau_{e}$ & \texttt{tau\_d} & 19 & \\ \hline
  Slope & $\alpha$ &\texttt{alpha} & 19 &  \\ \hline
  Brake position & $r$ & \emph{brake}& 20 &  \\ 
 \hhline{|====|}
\end{tabularx}









\section{Conclusion}
\label{sec:conclusion}

\section*{Acknowledgments}
\label{sec:acknowledgments}


\bibliographystyle{plainnat}  
\bibliography{lde-bib}
\end{document}

%%% Local Variables: 
%%% mode: latex
%%% TeX-master: t
%%% End: 
