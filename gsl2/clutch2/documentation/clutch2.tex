%%%%%%%%%%%%% preamble for all tex documents  %%%%
%%%%%%%%%%%%% current style defaults to scrartcl  but that could be changed
%%%%%%%%%%%%% with a script or by hand   
%%%%%%%%%%%%%
\documentclass[10pt,notitlepage,abstracton]{scrartcl}
\usepackage[a4, center, dvips]{crop}
\setlength{\linewidth}{\textwidth}
\usepackage{scrpage2}
\usepackage[sort,numbers]{natbib}
\usepackage{amsmath}
\usepackage{amscd}
\usepackage{amsthm}
\usepackage{amsfonts}
\usepackage{amsbsy}
\usepackage{amsxtra}
\usepackage{amssymb}
\usepackage[draft]{fixme}
\usepackage[utf8]{inputenc}
\usepackage{float}
\usepackage{url}
\usepackage{graphicx}
\usepackage{color}
\usepackage[ruled, section]{algorithm}
\usepackage{algpseudocode}
\usepackage{multirow}
\usepackage{tabularx}
\usepackage{subfigure}
\usepackage{tikz}
\usepackage{pgfplots,pgfplotstable}
\usepackage{gnuplot-lua-tikz}
\usepackage[normalem]{ulem}
\theoremstyle{plain}
\newtheorem{theorem}{Theorem}[section]
\newtheorem{lemma}[theorem]{Lemma}
\newtheorem{proposition}[theorem]{Proposition}
\newtheorem{corollary}[theorem]{Corollary}
\theoremstyle{plain}
\newtheorem{definition}{Definition}[section]
\newtheorem{example}{Example}[section]
\theoremstyle{plain}
\newtheorem*{remark}{Remark}
\newtheorem*{note}{Note}
\newtheorem{case}{Case}
\makeatletter
\renewcommand{\labelitemi}{$\m@th\triangleright$}
\renewcommand{\labelitemii}{$\m@th\cdot$}
\renewcommand{\labelitemiii}{\emdash}
\renewcommand{\labelitemiv}{\emdash\emdash}
\makeatother
\include{notation-macros}
\usepackage[scrpage]{svninfo} 
%%%%  not sure what to make of revision info right now
\title{Simple FMUs for clutches and gears}
\author{Claude Lacoursi{\`{e}}re \\
  HPC2N/UMIT, Ume{\aa} University\\
  SE-901 87, Ume{\aa}, Sweden\\
  \texttt{claude@hpc2n.umu.se}
}
\begin{document}
\svnInfo $Id$
\maketitle{}
\svnId
\begin{abstract}

\end{abstract}

\section{Introduction}
\label{sec:introduction}

This module models a clutch and a gear box with two rotational bodies
connected to each other by a piecewise linear force law.  The force can be
set to zero via a control parameter to simulate engagement and
disengagement.

The bodies themselves are connected to other modules via either
force-velocity, force-displacements, or displacements-displacement
couplings, depending on the chosen configuration.  All combinations are
supported.

The module can also compute the effective mobility so that it can be used
in a kinematic coupling configuration. 

\section{Internal forces}
\label{sec:internal}

We first consider internal forces between the two bodies.   

The case of the
gear with ratio $\alpha$ is a spring of the form
\begin{equation}
  \label{eq:1}
  f_{g} = 
  - K_{g}(\idx{\phi}{1} - \alpha\idx{\phi}{2})
\pm D_{g}(\idx{\omega}{1}  - \alpha\idx{\omega}{2}).
\end{equation}
This is the force on body $1$, and so $-f_{g}$ is applied to body $2$.
The reason for the first spring term is to avoid numerical slipping which
is important for the case of displacement coupling.  Because of this, we
the angles are reset when the gear.  

As for the clutch, we use a piecewise linear force law when it is engaged.
This models the small amount of flexibility necessary to absorb vibration
from the engine.  An ideal clutch should have a nonsmooth force law however
so that an arbitrary amount of torque could be delivered once fully
engaged, up to the slipping point.  This is not covered here.  The force
law is then
\begin{equation}
  \label{eq:2}
  f_{c} =
  \begin{cases}
     -K_{c_{1}} (\idx{\phi}{1} - \alpha\idx{\phi}{2})
     -D_{c_{1}}(\idx{\omega}{1}  - \alpha\idx{\omega}{2})  & \text{ for }
    \idx{\phi}{1}-\idx{\phi}{2}
    \in[\;\underline{\delta\phi},\overline{\delta\phi}\;] \\[0.5em]
     -K_{c_{2}} (\idx{\phi}{1} - \alpha\idx{\phi}{2})
     -D_{c_{2}}(\idx{\omega}{1}  - \alpha\idx{\omega}{2})  & \text{ for }
    \idx{\phi}{1}-\idx{\phi}{2}
    >\overline{\delta\phi} \\[0.5em]
     -K_{c_{3}} (\idx{\phi}{1} - \alpha\idx{\phi}{2})
     -D_{c_{3}}(\idx{\omega}{1}  - \alpha\idx{\omega}{2})  & \text{ for }
    \idx{\phi}{1}-\idx{\phi}{2}
    <\underline{\delta\phi}
  \end{cases}
\end{equation}
An example curve is shown below.   At present time, the clutch curve is set
directly in the source code and cannot be changed dynamically.


\pgfplotsset{every axis plot post/.append style={mark=none}}%standard value no mark
\pgfplotsset{ every mark/.append style={solid} }
\pgfplotsset{scaled y ticks=manual:{$10^{3}$}{\pgfmathparse{#1/1000}}}
\begin{tikzpicture}
  \begin{axis}[
    yticklabel style={%aligns the y ticks numbers automatically :-)
    /pgf/number format/fixed,
  %  /pgf/number format/precision=1,
  %  /pgf/number format/fixed zerofill,
  %  scaled y ticks=true
  }, 
  xticklabel style={%aligns the y ticks numbers automatically :-)
    /pgf/number format/fixed,
    /pgf/number format/precision=1,
    /pgf/number format/fixed zerofill
  },  
  xlabel=$\delta\phi$, ylabel=$\tau$, 
  xtick={-0.1, 0, 0.1, 0.2}]
  \addplot +[teal,line width=3pt] table {./clutch.dat};
  \end{axis}
\end{tikzpicture}

\section{Couplings}
\label{sec:couplings}

Several options are provided for coupling.  In the first case, we receive a
velocity from another module, and report a force back.  The force law is
linear
\begin{equation}
  \label{eq:4}
  \idxsub{f}{i}{p} = - \idxsub{K}{i}{p}\idx{\delta\phi}{i} -
  \idxsub{D}{i}{p}(\idx{v}{i} -
  \idx{\bar{v}}{e,i}), \text{ where } i = 1,2.
\end{equation}
In both cases, $-\idxsub{f}{i}{p}$ are reported as output to the respective
coupled module $(e,i)$.  

The module can also receive forces $\idxsub{f}{i}{e}$ at either port. Two
separate inputs are provided for the force.

The force inputs can be used for the case of kinematic couplings where a
global solver is responsible to compute pairwise forces necessary to
maintain algebraic conditions describing joints. 

\section{Equations of motion}
\label{sec:equations-motion-1}

Now that we have the forces, the equations of motion are simply
\begin{equation}
  \label{eq:5}
  \begin{split}
    \idx{\mathcal{J}}{1}\idx{\dot{\omega}}{1} & = c_{g}f_{g} + c_{c}f_{c} +
    \idxsub{f}{1}{p} + \idxsub{f}{1}{e} \\
    \idx{\mathcal{J}}{2}\idx{\dot{\omega}}{2} & = -c_{c}f_{g} + -c_{c}f_{c} +
    \idxsub{f}{2}{p} + \idxsub{f}{2}{e} \\
    \idx{\delta\dot{\phi}}{1} &= \idx{\omega}{1} - \idx{\bar{\omega}}{e,1} \\
    \idx{\delta\dot{\phi}}{2} & =\idx{\omega}{2} - \idx{\bar{\omega}}{e,2} 
  \end{split}
\end{equation}
and $\idx{\bar{\omega}}{e,i}$ are the current estimated values of the
angular velocities of the coupled system.   We also provide an option to
set
\begin{equation}
  \label{eq:6}
  \idx{\delta\phi}{i}  = \idx{\phi}{i} - \idx{\bar{\phi}}{e,i}
\end{equation}
for displacement-displacement coupling.



\section{Other outputs}
\label{sec:other-outputs}

This module can report two types of directional derivatives.  The first one
simply reports
\begin{equation}
  \label{eq:7}
  \fracpd{\idx{\dot{\omega}}{e,i}}{\idx{f}{e,i}} =
  \frac{1}{\idx{\mathcal{J}}{e,i}}. 
\end{equation}
The second one however reports the effective mobility over a communication
step, i.e., the numerical derivative 
\begin{equation}
  \label{eq:8}
  \mathcal{J}^{-1} = \fracpd{\idxsub{\dot{\omega}}{e,i}{k+1}}{\idxsub{f}{e,i}{k}} =
  \frac{1}{h}\frac{
    \idxsub{\bar{\omega}}{i,e}{k+1} -\idxsub{{\omega}}{i,e}{k+1}
  }{\delta \idx{f}{e,i}}, 
\end{equation}
where $k$ is the discrete time step, as this is needed for semi-implicit
time stepping used for kinematic couplings.  Here, quantities
$\idxsub{\omega}{e,i}{k+1}$ are the values obtained after integrating by a
small time $h$ for a given net force $f$, and values
$\idxsub{\bar{\omega}}{i,e}{k+1}$ are obtained after integrating by a small
time $h$ with input $f + \delta \idx{f}{e,i}$.  

The expression in Eqn.~\eqref{eq:8} is to be understood column-wise so that
\begin{equation}
  \label{eq:3}
  \begin{split}
    \mathcal{J}^{-1}_{{\bullet,e}} &= \frac{1}{h\norm{\delta \idx{f}{e}}}
    \begin{bmatrix}
      \idxsub{\bar{\omega}}{e}{k+1} -\idxsub{{\omega}}{e}{k+1} \\
      \idxsub{\bar{\omega}}{i}{k+1} -\idxsub{{\omega}}{i}{k+1} 
    \end{bmatrix}  
    \text{ and } \\
    \mathcal{J}^{-1}_{{\bullet,i}} &= \frac{1}{h\norm{\delta \idx{f}{i}}}
    \begin{bmatrix}
      \idxsub{\bar{\omega}}{e}{k+1} -\idxsub{{\omega}}{e}{k+1} \\
      \idxsub{\bar{\omega}}{i}{k+1} -\idxsub{{\omega}}{i}{k+1} 
    \end{bmatrix}  
  \end{split}
\end{equation}

Because there is only force coupling between the elements, the mobility
matrix is diagonal and the values reported are just
$1/\idx{\mathcal{J}}{1}$ and $\idx{\mathcal{J}}{2}$.

\begin{anfxnote}{data!}
  Should have data here which shows how the mobility values differ when
  different steps are used.  Also, this would show that the overall
  mobility matrix does depend on the spring constants.  


  Experiments to perform on the clutch. 
  \begin{itemize}
  \item  With all $0$ as init condition, show how mobility changes as a
    function of step and demonstrate convergence. 
  \item Show variations with strength of input and output springs 
  \item Show that low stiffness leads to either negative mobility or
    off-diagonal terms. 
  \item Show that the converged values are correct, i.e., with increasing
    stiffness, the system behaves as a whole. 
  \item Redo these experiments so that they are in the 4 different regimes,
    or rewrite the dynamics accordingly.  Easier to do last? 
  \end{itemize}


  For the gears, just run standard experiments.  Figure out why, oh why, we
  can get correct values with standard stepping matrices. 

  Add plots of the dynamics of the clutch to show model limitations. 

\end{anfxnote}

\section{Jacobians}
\label{sec:jacobians}

The piecewise linear clutch model requires a very high damping constant in
order to keep the angle difference within reasonable bounds so this is best
integrated using an implicit integrator which requires a Jacobian matrix. 


\section{Inputs and outputs}
\label{sec:inputs-outputs}

The list of parameters is as follows.
\begin{tabularx}{1.0\linewidth}[H]{ ||X|c|c|c|c|c|| }
  \hhline{|======|}
  Name & Symbol & Variable & Value ref. & Range & Default \\
  \hhline{|======|}
  Initial angle & $\idx{\phi}{1}(0)$ & \texttt{\footnotesize{x0\_e}}&  0 & & \\ \hline
  Initial velocity & $\idx{\omega}{1}(0)$ & \texttt{\footnotesize{v0\_e}}& 1 & & \\ \hline
  Initial angle diff& $\idx{\delta\idx{\phi}{1}}(0)$ & \texttt{\footnotesize{dx0\_e}}& 1 & & \\ \hline
  Initial angle & $\idx{\phi}{2}(0)$ & \texttt{\footnotesize{x0\_s}}&  0 & & \\ \hline
  Initial velocity & $\idx{\omega}{2}(0)$ & \texttt{\footnotesize{v0\_s}}& 1 & & \\ \hline
  Initial angle diff& $\idx{\delta\idx{\phi}{2}}(0)$ & \texttt{\footnotesize{dx0\_s}}& 1 & & \\ \hline
  Coupling spring & $K_{p_{1}}$ & \texttt{\footnotesize{k\_ec}}& 1 & & \\ \hline
  Coupling spring & $D_{c_{1}}$ & \texttt{\footnotesize{gamma\_ec}}& 1 & & \\ \hline
  Integration & $\int\dd\delta\idx{\dot{\phi}}{1}$ & \texttt{\footnotesize{integrate_dx_e}}& 1 & & \\ \hline
  Coupling spring & $K_{p_{2}}$ & \texttt{\footnotesize{k\_sc}}& 1 & & \\ \hline
  Coupling spring & $D_{p_{2}}$ & \texttt{\footnotesize{gamma\_sc}}& 1 & & \\ \hline
  Integration & $\int\dd\delta\idx{\dot{\phi}}{2}$ & \texttt{\footnotesize{integrate_dx_s}}& 1 & & \\ \hline
  Mass in shaft& $\idx{\mathcal{J}}{1}$ & \texttt{\footnotesize{mass\_e}}& 1 & & \\ \hline
  In shaft damping & $D_{1}$ & \texttt{\footnotesize{gamma\_e}}& 1 & & \\ \hline
  Mass out shaft& $\idx{\mathcal{J}}{2}$ & \texttt{\footnotesize{mass_s}}& 1 & & \\ \hline
  Out shaft damping & $D_{2}$ & \texttt{\footnotesize{gamma\_s}}& 1 & & \\ \hline
  Clutch damping& $D_{c}$ & \texttt{\footnotesize{gamma\_s}}& 1 & & \\ \hline
  
  EPCE filter &  & \texttt{\footnotesize{filter\_length}} & 98 &
  \hhline{|======|}
\end{tabularx}


\bibliographystyle{plainnat}  
\bibliography{lde-bib}



\end{document}

%%% Local Variables: 
%%% mode: latex
%%% TeX-master: t
%%% End: 
