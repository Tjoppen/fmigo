%%%%%%%%%%%%% preamble for all tex documents  %%%%
%%%%%%%%%%%%% current style defaults to scrartcl  but that could be changed
%%%%%%%%%%%%% with a script or by hand   
%%%%%%%%%%%%%
\documentclass[10pt,notitlepage,abstracton]{scrartcl}
\usepackage[a4, center, dvips]{crop}
\setlength{\linewidth}{\textwidth}
\usepackage{scrpage2}
\usepackage[sort,numbers]{natbib}
\usepackage{amsmath}
\usepackage{amscd}
\usepackage{amsthm}
\usepackage{amsfonts}
\usepackage{amsbsy}
\usepackage{amsxtra}
\usepackage{amssymb}
\usepackage[draft]{fixme}
\usepackage[utf8]{inputenc}
\usepackage{float}
\usepackage{url}
\usepackage{graphicx}
\usepackage{color}
\usepackage[ruled, section]{algorithm}
\usepackage{algpseudocode}
\usepackage{multirow}
\usepackage{tabularx}
\usepackage{subfigure}
\usepackage{tikz}
\usepackage{pgfplots,pgfplotstable}
\usepackage{gnuplot-lua-tikz}
\usepackage[normalem]{ulem}
\theoremstyle{plain}
\newtheorem{theorem}{Theorem}[section]
\newtheorem{lemma}[theorem]{Lemma}
\newtheorem{proposition}[theorem]{Proposition}
\newtheorem{corollary}[theorem]{Corollary}
\theoremstyle{plain}
\newtheorem{definition}{Definition}[section]
\newtheorem{example}{Example}[section]
\theoremstyle{plain}
\newtheorem*{remark}{Remark}
\newtheorem*{note}{Note}
\newtheorem{case}{Case}
\makeatletter
\renewcommand{\labelitemi}{$\m@th\triangleright$}
\renewcommand{\labelitemii}{$\m@th\cdot$}
\renewcommand{\labelitemiii}{\emdash}
\renewcommand{\labelitemiv}{\emdash\emdash}
\makeatother
\include{notation-macros}
\usepackage[scrpage]{svninfo} 
%%%%  not sure what to make of revision info right now
\title{Trailer model}
\author{Claude Lacoursi{\`{e}}re \\
  HPC2N/UMIT, Ume{\aa} University\\
  SE-901 87, Ume{\aa}, Sweden\\
  \texttt{claude@hpc2n.umu.se}
}
\begin{document}
\svnInfo $Id$
\maketitle{}
\svnId
\begin{abstract}

\end{abstract}

\section{Introduction}
\label{sec:introduction}

This is a simple model for a ``trailer'', a point mass moving along a
straight line but with variable elevation.   This is intended to perform
simple stability tests for cosimulation algorithms, especially for the case
where different simulations involve objects of different masses. 

\section{The model}
\label{sec:model}

We represent the trailer as a point mass $m$ moving in a straight line. Even
though we are considering the trailer to have both vertical and
longitudinal motion, we take $x\in\R$ as the distance traveled along the road
which has variable grade $\alpha(x)$.  


The trailer is powered via a shaft with angle and angular speed
$\idx{\phi}{i},\idx{\omega}{i}$.  This shaft goes is connected to the axle
with a gear with ratio $\idx{r}{d}$ from the differential.  Assuming no
slip, the speed of the point mass is related to that of the wheel via
$\dot{x} = \idx{r}{w}\idx{\omega}{w}$.

The relevant portion of the gravity force is
\begin{equation}
  \label{eq:1}
  \idx{f}{g} = - m g \sin(\alpha ( x ) ).
\end{equation}
A drag force for air resistance is introduced as
\begin{equation}
  \label{eq:2}
  \idx{f}{d} = - \sign(\dot{x}) \frac{1}{2}\rho A C_{d} \dot{x}^{2},  
\end{equation}
rolling resistance is then
\begin{equation}
  \label{eq:3}
 \idx{f}{r} = - \sign(\dot{x})(c_{r_{1}}\abs{\dot{x}} + c_{r_{0}}) m g \cos(\alpha(x))
\end{equation}
and a braking force
\begin{equation}
  \label{eq:4}
  \idx{f}{b} = - r \mu m g \cos(\alpha(x)), \text{ where }   r\in[0,1].
\end{equation}
As for the input forces, we either accept a torque directly in which case
\begin{equation}
  \label{eq:5}
  \idx{f}{d} = \frac{1}{\idx{r}{d}\idx{r}{w}}\idx{\tau}{i}, 
\end{equation}
or we use a coupling spring with
\begin{equation}
  \label{eq:6}
  \begin{split}
    \idx{f}{d} &= \frac{1}{\idx{r}{w}} \left( -\idx{k}{i} \delta -
      \idx{\gamma}{i}( \frac{\idx{r}{d}\dot{x}}{\idx{r}{w}} - \idx{\omega}{i}
      )
    \right) \\
    \dot{\delta} &= \frac{\idx{r}{d}\dot{x}}{\idx{r}{w}} - \idx{\omega}{i}
  \end{split}
\end{equation}
The 






\section{Conclusion}
\label{sec:conclusion}

\section*{Acknowledgments}
\label{sec:acknowledgments}


\bibliographystyle{plainnat}  
\bibliography{lde-bib}
\end{document}

%%% Local Variables: 
%%% mode: latex
%%% TeX-master: t
%%% End: 
